%%%%%%%%%%%%%%%%%%%%%%%%%%%%%%%%%%%%%%%%%
% University/School Laboratory Report
% LaTeX Template
% Version 2.0 (4/12/12)
%
% This template has been downloaded from:
% http://www.latextemplates.com
%
% License:
% CC BY-NC-SA 3.0 (http://creativecommons.org/licenses/by-nc-sa/3.0/)
%
% Original header:
%
%
%%%%%%%%%%%%%%%%%%%%%%%%%%%%%%%%%%%%%%%%%

%----------------------------------------------------------------------------------------
%	DOCUMENT CONFIGURATIONS
%----------------------------------------------------------------------------------------

\documentclass{article}

\usepackage{graphicx} % Allows the inclusion of images

\title{How to Install CyNEST} % Title

\author{Jonny \textsc{Quarta}} % Author name

\date{\today} % Specify a date for the report

\begin{document}

\maketitle % Insert the title, author and date


\setlength\parindent{0pt} % Removes all indentation from paragraphs

\renewcommand{\labelenumi}{\alph{enumi}.} % Make numbering in the enumerate environment by letter rather than number (e.g. section 6)

%----------------------------------------------------------------------------------------
%	SECTION 1
%----------------------------------------------------------------------------------------


In order to correctly install CyNEST, you must achieve the following very simple operations :
\\ \\
\begin{itemize}
\item Place your terminal in the \emph{nest-cynest} folder (where the NEST code is situated).

\item Execute \textbf{./bootstrap.sh}.\\ This will run the automake, autoconf and libtoolize tools, generating the \emph{configure} file.

\item Leave the current directory and create another one (the name is ininfluent), then enter this new folder.

\item Call the \emph{configure} file in the \emph{nest-cynest} folder.\\Hence type \\ \textbf{../nest-cynest/configure --prefix="install-dir" - -enable-cynest}\\ in your terminal, where \emph{install-dir} is the directory where you want to install CyNEST.

\item Type \textbf{make -j4} and then \textbf{make install} in the terminal. This will compile and install CyNEST on the computer. 

\item Configure the paths in the terminal. You can do that by modifying, for example, the \emph{/home/"user"/.bashrc} file. Add the following lines:
\begin{verbatim}
export PATH="[installation path of CyNEST]/bin:$PATH"

export PYTHONPATH="[installation path of CyNEST]/lib/
                    python2.7/site-packages:$PYTHONPATH"
\end{verbatim}

\item It is now possible to load python and, by typing \textbf{import cynest} in the python terminal, CyNEST can be used.


\end{itemize}

\end{document}