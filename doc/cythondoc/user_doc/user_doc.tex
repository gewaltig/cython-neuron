%%%%%%%%%%%%%%%%%%%%%%%%%%%%%%%%%%%%%%%%%
% University/School Laboratory Report
% LaTeX Template
% Version 2.0 (4/12/12)
%
% This template has been downloaded from:
% http://www.latextemplates.com
%
% License:
% CC BY-NC-SA 3.0 (http://creativecommons.org/licenses/by-nc-sa/3.0/)
%
% Original header:
%
%
%%%%%%%%%%%%%%%%%%%%%%%%%%%%%%%%%%%%%%%%%

%----------------------------------------------------------------------------------------
%	DOCUMENT CONFIGURATIONS
%----------------------------------------------------------------------------------------

\documentclass{article}

\usepackage{graphicx} % Allows the inclusion of images

\title{Writing Customable Cython Neurons \\ User Documentation} % Title

\author{Jonny \textsc{Quarta}} % Author name

\date{\today} % Specify a date for the report

\begin{document}

\maketitle % Insert the title, author and date


\setlength\parindent{0pt} % Removes all indentation from paragraphs

\renewcommand{\labelenumi}{\alph{enumi}.} % Make numbering in the enumerate environment by letter rather than number (e.g. section 6)

%----------------------------------------------------------------------------------------
%	SECTION 1
%----------------------------------------------------------------------------------------
\section{Introduction}
CyNEST proposes the possibility to create custom neuron models in cython, without any knowledge of C++. The first important thing to know is that for correct behaviour, at least Cython 0.18 must be installed.\\
For any information about Cython, from the installation to the language syntax, please consult the home page (www.cython.org).\\

The main strength of this new feature is the conjunction of use easiness with very little performance loss.\\

\section{Writing a new neuron}
This section deals with the basic operations one has to perform in order to write a custom neuron. The neuron class has to follow a certain structure, which will be explained during the construction of a little example. \\
Let assume we want to create a new custom neuron, which we will call \textbf{test\_neuron}. Then we create a new file, \emph{test\_neuron.pyx}. \textbf{Note that the file name and the neuron name have to be the same}, (in this case test\_neuron). Once the file is created, we can put our skeleton code:
\begin{verbatim}
include "[installation path]/include/Neuron.pyx"

cdef class test_neuron(Neuron):
    cdef ....
    
    def __cinit__(self):
        pass

    cdef calibrate(self):
        pass

    cdef setStatus(self, params):
        pass

    cdef getStatus(self):
        pass

    cdef update(self):
        pass
\end{verbatim}
As we can see, the test\_neuron class extends from Neuron. \textbf{This is mandatory}. Also note that the Neuron base class is contained in a special file located at \emph{[installation path]/include/Neuron.pyx}, where the installation path is the path where CyNEST has been installed.\\
In addition, some methods can be overwritten (these are situated in the base class Neuron). Method overwriting is optional, however at least calibrate and update should be changed, otherwise the neuron will be useless.\\ \\
Here's a brief description of every method :
\begin{itemize}
\item \_\_cinit\_\_ : this is the constructor. Here should be initialized the totality of internal members the neuron will use. Note that since this is a cython class, one cannot just declare a new member, as these new objects are not dinamically attached to the class. In order to declare a new member, one has to make a \emph{cdef ...} outside the methods. 
\item getStatus : this method is called whenever the user calls the CyNEST function \textbf{ cynest.GetStatus(...)} in the python terminal (or NEST internally calls it). Note that it should output a dictionary containing the totality of the members the user created.
\item setStatus : this method is called whenever the user calls the CyNEST function \textbf{ cynest.SetStatus(...)} or \textbf{cynest.SetDefaults(...)} in the python terminal (or NEST internally calls it). This could be usefull if the neuron has to adapt some other members with respect to special ones. Note that it takes as argument a dictionary containing the totality of the updated paramaters the user created. The user should update the model with them.
\item calibrate : this method is called before the simulation start. Its purpose is to adjust the various neuron parameters before the simulation.
\item update : this is the most important method and is called repeatedly during a simulation. Its purpose is to update the various neuron parameters according to the inputs (everything will be explained later).
\end{itemize}
Note that the way the user handles the model parameters is completely unimportant to NEST, as long as they are correctly interfaced with the system (using the \emph{setStatus} and \emph{getStatus} methods).


\section{Restrictions}
As everything on Earth, nothing is perfect and few restrictions have to be followed in order to write compilable code. Here's the list:
\begin{itemize}
\item Some special neuron members are already defined, their type cannot be changed and their values should not be updated (not always true, see next section).
\item If the neuron declares a new member of custom type (maybe defined in another file), it should not be put in the dictionary created during the \emph{getStatus} method. This "makes" the parameter private and CyNEST will not try to convert it from Python to C++ (custom objects cannot be translated into C++ objects). However, if the member has a builtin type (integer, boolean, double, dictionary, list, etc...), there is no problem.
\item Every other cython limitation also applies.
\end{itemize}

\section{Standard Parameters and Special Functions}
This section deals with the communication between the custom neuron and other ones.\\
As one could imagine, not everything can be done when writing a custom neuron with respect to writing a static C++ one. As an example, events cannot be accessed (SpikeEvent, CurrentEvent, etc...). However, these situations are handled in a very general way, providing almost the same confort as normal neurons do.
\subsection*{Standard Parameters}
During a model simulation, the communication between the custom neuron and the kernel is very important. In particular, the kernel should know, at least when he needs, the value of the different neuron members. However, during the actual simulation only a few members are important to NEST and these are called Standard Parameters.
These have been declared and are updated before every call to the \emph{update} method. Here's the list:
\begin{itemize}
\item ex\_spikes : this is the number of excitatory spikes incoming at a certain point. Note that this is a double value, because every spike in automatically multiplied by it's weight, which could be a real value.
\item in\_spikes : this is the number of inhibitory spikes incoming at a certain point. Note that this is a double value, because every spike in automatically multiplied by it's weight, which could also be a real value. Also note that the value is always negative, because the weight are. If one want to sum all the spikes, he should write \emph{ex\_spikes + in\_spikes} and not \emph{ex\_spikes - in\_spikes}.
\item currents : this is the currents upcoming at a certain point in time. This is also a double value.
\item t\_lag : this integer value represents the current simulation point in time (this would be the t variable of a function).
\item spike : this is the only modifiable standard parameter. That's the neuron output and is an integer value (not a boolean one). Set it at 1 if the neuron spikes and 0 otherwise.
\end{itemize}
All these Standard Parameters are contained in the base Neuron class and, since their behaviour is highly optimized, provide a very little performance loss compared to a native neuron. They do not need to be included in the \emph{setStatus} and \emph{getStatus} methods.


\subsection*{Special Functions}
When writing a neuron, one could need some other special parameters concerning the way the simulation has been set. In a normal C++ neuron, the totality of those values would be accessible from the Time and Scheduler classes (for more information about these classes, see the NEST official documentation), but these classes are not visible from the custom neuron side. 
There is however a way in order to acces the most important values and the base Neuron class provides an object, \emph{time\_scheduler}, which in turn contains the entire list of methods (15 in total):
\begin{itemize}
\item get\_ms\_on\_resolution()

\item get\_ms\_on\_tics(tics)

\item get\_ms\_on\_steps(steps)

\item get\_tics\_on\_resolution(self)

\item get\_tics\_on\_steps(steps)

\item get\_tics\_on\_ms(ms)
   
\item get\_tics\_on\_ms\_stamp(ms\_stamp)

\item get\_steps\_on\_resolution()

\item get\_steps\_on\_tics(tics)

\item get\_steps\_on\_ms(ms)

\item get\_steps\_on\_ms\_stamp(ms\_stamp)

\item get\_modulo(value)

\item get\_slice\_modulo(value)

\item get\_min\_delay()

\item get\_max\_delay()

\end{itemize}

As an example, writing \emph{self.time\_scheduler.get\_ms\_on\_resolution()} is the same as writing (in the NEST code, C++) \emph{Time::get\_resolution().get\_ms()} and \\ \emph{self.time\_scheduler.get\_steps\_on\_ms(ms)} the same as \emph{Time(Time::ms(ms)).get\_steps()}. In addition, writing \emph{self.time\_scheduler.get\_min\_delay()} could be compared to the C++ \emph{Scheduler::get\_min\_delay()}.\\
Note that calling these functions takes much time because the call spreads through many other methods in order to achieve the Time or Scheduler classes. It is therefore advised to avoid these calls in the update method and to create variables instead.

\section{Using the new neuron}
After the custom neuron has been written, it can be simply imported and used in the python terminal. In order to achieve that, the user has to use the \emph{cynest.RegisterNeuron(...)} method and then use it like a normal one. The .pyx file should be placed in the same location as the python terminal has been invoked and no file extension has to be written (just like a normal python import).\\
Note that when registering the neuron, the pyx file is automatically compiled and linked, so compilation errors could occur. In this case, just check them and fix the neuron model.

\section{Performances}
The system has been tested during a single-threaded simulation of 40 ms concerning over 1000 randomly connected neurons and compared to their native and SLI counterparts. The results are:
\begin{itemize}
\item Native neuron: realtime factor is 0.3254
\item SLI neuron: realtime factor is 0.0046
\item Cython neuron: realtime factor is 0.1047
\end{itemize}
We can therefore conclude that the Cython neuron is only about 3 times slower than the native version, whereas the SLI neuron is up to 70 times slower.\\
Note that during different test trials, values could slightly change because of the random factor introduced in the network.\\

\end{document}